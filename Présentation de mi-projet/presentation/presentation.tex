\documentclass{beamer}

\usepackage[francais]{babel}
\usepackage[T1]{fontenc}
\usepackage[utf8]{inputenc}
\usepackage{beamerthemecs}
\usepackage{beamerouterthemecs}
\usepackage{beamerfontthemecs}
\usepackage{beamerinnerthemecs}
\usepackage{beamercolorthemecs}

\usetheme{cs}
\useoutertheme{cs}
\usefonttheme{cs}
\useinnertheme{cs}
\usecolortheme{cs}

\title[Visualisation de traces réseaux]{Visualisation de traces réseaux}
\author{\textbf{Thibault \textsc{Lengagne}et Nicolas \textsc{Ngô-Maï}}}
\institute{Centrale Supélec - Campus de Rennes}

\begin{document}

  \begin{frame}
    \titlepage
  \end{frame}
  
  \AtBeginSection[] {
    \begin{frame}
      \frametitle{Plan}
      \tableofcontents[currentsection, hideothersubsections, pausesubsections]
    \end{frame} 
  }

 \section{Sujet - Objectifs - Choix}
  \begin{frame}
   \frametitle{Rappel du Sujet}
   Créer un outils pour pentester, permettant d'analyser efficacement une trace réseau
   \begin{itemize}
    \item Les adresse IP identifiées ( + géolocalisation, résolution DNS)
    \item Les protocoles utilisés ( en particulier non chiffrés ou mal configurés)
    \item Les noeuds importants du réseau (serveur de fichier, DNS, LDAP...)
    \item Extraire des traces réseaux filtrées, extraire les données sensibles
   \end{itemize}
  \end{frame}
  
  \begin{frame}
   \frametitle{Retour sur les choix technologique}
   Nous voulions interfacer plusieurs outils (Ettercap, ChaosReader, tcptrace...)
   -> Finalement, nous utilisons uniquement Scapy
   Pour stocker les résultats, la manipulation des JSON étant laborieuse, nous avons choisi d'utiliser une base de donnée PostGreSQL
   Nous avons trouvé un outil très puissant de visualisation : D3.js
   -> Nous avons donc créer une interface web (Flask, SQLAlchemy) connectée à PostgreSQL
  \end{frame}
  
  \begin{frame}
   Schéma fonctionnel :
  \end{frame}

  \begin{frame}
     \frametitle{Retour sur les objectifs fixés}
    Nous avons remplis les objectifs suivants :
    \begin{itemize}
     \item Extraction de sessions, des utilisateurs, et des protocoles
     \item Création de statistiques, extraction des données http
     \item Visualisation en barre parallèle
    \end{itemize}
  \end{frame}
  
 \section{Démonstration}
  \begin{frame}
    \frametitle{Lancement de l'outil}
    Le projet est disponible sur https://github.com/lechinoix/Pcap-visualization-project
    Après avoir suivi la procédure d'installation, on démarre le serveur pour accéder à l'interface web
    Il suffit de disposer d'un fichier .pcap à analyser
  \end{frame}

  %Ici une démonstration live de 5 min

  \begin{frame}
    \frametitle{Question, remarques ?}
  \end{frame}

  \section{Travail à venir}
  \begin{frame}
    Nous devons encore remplir ces objectifs
    \begin{itemize}
     \item Ajouter les différents filtres possibles
     \item Extraire d'un pcap a partir des trames filtrées
     \item Extraire les données non chiffrées des protocoles SMTP,IMAP,POP,LDAP
     \item Ajouter la résolution DNS
     \item Ajouter d'autres mode de visualisation (carte des IP,..)
    \end{itemize}
  \end{frame}


  \section{Conclusion}
  \begin{frame}
    \begin{center}
      Merci de votre attention !
    \end{center}
  \end{frame}

\end{document}
